\documentclass[a4paper, 12pt]{article}

% Pacotes:
\usepackage[brazilian]{babel}
\usepackage[utf8]{inputenc}
\usepackage[T1]{fontenc}
\usepackage[margin=2cm]{geometry}
\usepackage{indentfirst}
\usepackage{hyperref}

% Comandos:
\newcommand{\ordm}{\textsuperscript{\underline{o}}}
\newcommand{\ordf}{\textsuperscript{\underline{a}}}
\newcommand{\utf}{Universidade Tecnológica Federal do Paraná - Câmpus Toledo}
\newcommand{\tcc}{Trabalho de Conclusão de Curso}

\title{EDITAL de TCC1 - ADNP\\\textbf{Engenharia Eletrônica}}
\date{***}
\author{Universidade Tecnológica Federal do Paraná - Câmpus Toledo\\Coordenação do curso de Engenharia Eletrônica}


\begin{document}
    \maketitle
    
    O Professor Responsável pela Atividade de \tcc{ } (TCC) de Engenharia Eletrônica da \utf, no uso de suas atribuições estabelecidas pela Resolução N\textsuperscript{o} 018/18 – COEPP, de 11 de abril de 2018, torna público o cronograma de atividades de TCC1 para o período das Atividades Didáticas não Presenciais (ADNP).

    \section{Medidas necessárias devido à pandemia do COVID-19}

    Considerando a pandemia de COVID-19 e a Instrução Normativa N\textsuperscript{o} 08, de 11 de julho de 2020 - UTFPR todas as defesas de TCC de que tratam esse edital devem ser feitas de forma remota através de videoconferência. A ferramenta de reunião digital a ser utilizada é de responsabilidade do(a) Orientador(a).
    
    O preenchimento de todos os formulários citados pode ser feito de forma digital, com inclusão de assinatura escaneada, ou assinado manualmente para posterior digitalização.

    \section{Cronograma}
    \label{sec:CRO}
    
    Os procedimentos para a disciplina de TCC1 seguirão o seguinte cronograma:
    \begin{itemize}
    	\item \textbf{Início das ADNPs}: 03/08/2020;
    	\item \textbf{Prazo final para entrega do termo de orientação no Moodle}: 18/08/2020;
    	\item \textbf{Período de agendamentos das defesas}: 20/08/2020 à 31/10/2020;
    	\item \textbf{Prazo para a entrega da versão final}: 05/12/2020.
    \end{itemize}
    
    \section{Moodle e Fomulários}
    
    Um curso foi criado no Moodle para a entrega das  documentações exigidas. O discente pode fazer sua matrícula através da chave de acesso que será entregue pelo professor responsável pela disciplina. Em caso de problemas na inscrição, o professor responsável deverá ser consultado através do e-mail pfrimer@utfpr.edu.br. Acesse a página do curso \href{https://moodle.td.utfpr.edu.br/course/view.php?id=507}{clicando aqui}.
    
    Todos os formulários citados neste edital podem ser baixados na página da disciplina, sendo que as instruções de preenchimento também se encontram no Moodle.
    
    \section{Termo de orientação e TCC em empresa}
    
    Para formalizar a orientação do TCC, os discentes matriculados em TCC1 devem entregar o Termo de Orientação (Formulário 1), que deve ser preenchido, assinado pelo orientador e escaneado. O orientador deve manter a cópia original. A cópia digitalizada deve ser entregue através do Moodle no prazo estabelecido no cronograma (Seção \ref{sec:CRO}).
    
    Caso TCC seja realizado em uma empresa, o Termo de Compromisso de Realização de TCC em Empresa (Formulário 7) também deve ser entregue junto com o termo de orientação.

    \section{Agendamento da defesa de TCC1}

    O agendamento da defesa de TCC1 deverá ser formalizado através do Moodle (requisite a chave de inscrição na disciplina com o professor responsável). A documentação necessária é:
    
    \begin{itemize}
    	\item Uma cópia digital do projeto de TCC;
    	\item Autorização do orientador para a apresentação do Seminário de TCC 1 (Formulário 3);
    \end{itemize}
		
    Observações importantes:
    
    \begin{itemize}
    	\item As apresentações de TCC1 devem, obrigatoriamente, ocorrerem através de videoconferência, seguindo as normas da Universidade para o combate ao COVID-19.
    	\item A entrega do texto para a Banca deve ser feita com antecedência mínima de 7 (sete) dias antes da defesa;
    	\item Os discentes ou os orientadores são responsáveis por entregar a cópia digital ou impressa do TCC para a banca;
    	\item A data de apresentação informada no formulário 3 devem ser previamente agendada com a banca pelo(a) Orientador(a);
    	\item Mudanças relativas à data de apresentação e a composição dos membros da banca deverão ser informadas e justificadas pelo(a) orientador(a);
    	\item O período de agendamento para os trabalhos de TCC1 está exposto na Seção \ref{sec:CRO};
    \end{itemize}

	\section{O seminário de TCC1}
	
	O(a) presidente da banca é o orientador(a), que deverá coordenar as etapas de apresentação. Esta é divida em três períodos: defesa; arguição e deliberação. 
	
	Os candidatos terão entre 10 e 15 minutos para realizar a defesa. Logo após, cada membro da banca terá de 10 a 15 minutos para apresentar suas correções e questionamentos. Por fim, os membros da banca devem se reunir de forma reservada para deliberar o resultado.
	
	Cada membro da banca deve receber um cópia digital da Ficha de Avaliação de TCC 1 (Formulário 4). O preenchimento pode ser feito de forma digital, com inclusão de assinatura escaneada, ou assinado manualmente para posterior digitalização. \textbf{As fichas preenchidas não devem ser apresentadas aos candidatos}.
	
	O(a) orientador(a) deve entregar as fichas de avaliação preenchidas para o professor responsável através do e-mail pfrimer@utfpr.edu.br.
	
	A nota final será composta por: 20\% da nota do orientador e 40\% de cada membro.
	
	O preenchimento de uma ATA de defesa é opcional.
		
	\section{Divulgação dos resultados}
	
	As notas serão divulgadas pelos orientadores ou posteriormente no sistema acadêmico.
	
	\section{Entrega da versão final de TCC1}
	
	A versão final do projeto de TCC deverá ser entregue através do Moodle, em formato digital (PDF), para o Professor Responsável pela Atividade de TCC até a data estipulada na seção \ref{sec:CRO}. Junto como o texto, também deve ser entregue:
	
	\begin{itemize}
		\item Formulário de entrega de versão final (Formulário 6);
		\item Formulário de Acompanhamento das Atividades (Formulário 2).
	\end{itemize}
	
	\section{Observações}
	
	\begin{itemize}
		\item Os textos deverão ser elaborados segundo as normas da ABNT;
		
		\item Os alunos matriculados em TCC1 e que não obtiverem aprovação podem fazer uma nova tentativa, se houver tempo hábil para uma segunda apresentação;
		
		\item O pedido para uma nova banca só pode ser feito 20 dias antes de terminar o período da ADNP;
		
		\item As propostas podem ser apresentadas em equipe, conforme o regulamento do TCC, com até três componentes.
	\end{itemize}


	\section{Casos omissos}

	Casos omissos à esse edital serão decididos pela coordenação do curso em conjunto com o professor responsável pela atividade de TCC.
	

	
	
	
	
	
	
	
	

    

\end{document}